\documentclass{article}

\usepackage[utf8]{inputenc}
\usepackage[T1]{fontenc} % Ajouté pour éviter les problèmes d'encodage avec le français
\usepackage{graphicx}
\usepackage[margin=1in, bottom=0.5in]{geometry}
\usepackage{lipsum}
\usepackage{tabto}
\usepackage[french]{babel}
\usepackage{helvet} % Ajouté pour utiliser la police Helvetica
\renewcommand{\familydefault}{\sfdefault} % Définit Helvetica comme police par défaut


\setlength{\parindent}{0pt}
\setlength{\parskip}{\baselineskip}
\pagestyle{empty}% Remove any headers/footers

% Nouvelle commande pour lire un fichier texte et échapper les caractères spéciaux
\newcommand{\readtextfile}[1]{%
    \begingroup
    \catcode`\#=12 % Make # a non-special character
    \catcode`\$=12 % Make $ a non-special character
    \catcode`\_=12 % Make _ a non-special character
    \catcode`\&=12 % Make & a non-special character
    \catcode`\~=12 % Make ~ a non-special character
    \catcode`\%=12 % Make % a non-special character
    \catcode`\^=12 % Make ^ a non-special character
    \input{#1}%
    \endgroup
}

\begin{document}
\sffamily% Default font for this letter

\newcommand{\readdestfile}[1]{
    \newread\file
    \openin\file=#1
    \begingroup
    \obeylines
    \everypar{\tabto{30em}}
    \catcode`\#=12 % Make # a non-special character
    \catcode`\$=12 % Make $ a non-special character
    \catcode`\_=12 % Make _ a non-special character
    \catcode`\&=12 % Make & a non-special character
    \catcode`\~=12 % Make ~ a non-special character
    \catcode`\%=12 % Make % a non-special character
    \catcode`\^=12 % Make ^ a non-special character
    \loop\unless\ifeof\file
        \read\file to \line
        \ifeof\file\else\line\par\fi
    \repeat
    \closein\file
    \endgroup
}

% Information de l'expéditeur et date sur la même ligne
Carlos Antonio Reyes Peña \\
Av. C.-F.- Ramuz 3
1009 Pully \\
Téléphone : 078 914 93 64 \\
carlosantonio.reyespena@gmail.com



% Information du destinataire avec tabulation

\begingroup
\parskip=0pt % Remove additional space between paragraphs within this group
\readdestfile{Compilation/destinataire.txt}
\endgroup

\bigskip

\tabto{30em} À Pully, le \today \\


% Sujet
\vspace*{0.6in} % Ajustez cet espace pour positionner le sujet à la hauteur désirée
\textbf{\readtextfile{Compilation/sujet.txt}}
\vspace{0.1\baselineskip} % Ajustez l'espace entre le sujet et la ligne
\hrule height 0.5pt % Ajustez l'épaisseur de la ligne



% Contenu de la lettre
\readtextfile{Compilation/corp.txt} % Utiliser la nouvelle commande


\bigskip
\bigskip


\tabto{30em} Carlos Antonio Reyes Peña
\bigskip
\bigskip
\bigskip

% Signature
\tabto{30em}\includegraphics[height=4\baselineskip]{signature.png} % Remplacez par l'image de votre signature si vous en avez une

\vfill % Pushes the following content to the bottom

\begin{minipage}[t]{\textwidth}
\textit{Annexes: CV, diplômes}
\end{minipage}

\end{document}
